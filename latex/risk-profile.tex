\documentclass[11pt]{article}

    
\usepackage[breakable]{tcolorbox}
    \usepackage{parskip} % Stop auto-indenting (to mimic markdown behaviour)
    

    % Basic figure setup, for now with no caption control since it's done
    % automatically by Pandoc (which extracts ![](path) syntax from Markdown).
    \usepackage{graphicx}
    % Keep aspect ratio if custom image width or height is specified
    \setkeys{Gin}{keepaspectratio}
    % Maintain compatibility with old templates. Remove in nbconvert 6.0
    \let\Oldincludegraphics\includegraphics
    % Ensure that by default, figures have no caption (until we provide a
    % proper Figure object with a Caption API and a way to capture that
    % in the conversion process - todo).
    \usepackage{caption}
    \DeclareCaptionFormat{nocaption}{}
    \captionsetup{format=nocaption,aboveskip=0pt,belowskip=0pt}

    \usepackage{float}
    \floatplacement{figure}{H} % forces figures to be placed at the correct location
    \usepackage{xcolor} % Allow colors to be defined
    \usepackage{enumerate} % Needed for markdown enumerations to work
    \usepackage{geometry} % Used to adjust the document margins
    \usepackage{amsmath} % Equations
    \usepackage{amssymb} % Equations
    \usepackage{textcomp} % defines textquotesingle
    % Hack from http://tex.stackexchange.com/a/47451/13684:
    \AtBeginDocument{%
        \def\PYZsq{\textquotesingle}% Upright quotes in Pygmentized code
    }
    \usepackage{upquote} % Upright quotes for verbatim code
    \usepackage{eurosym} % defines \euro

    \usepackage{iftex}
    \ifPDFTeX
        \usepackage[T1]{fontenc}
        \IfFileExists{alphabeta.sty}{
              \usepackage{alphabeta}
          }{
              \usepackage[mathletters]{ucs}
              \usepackage[utf8x]{inputenc}
          }
    \else
        \usepackage{fontspec}
        \usepackage{unicode-math}
    \fi

    \usepackage{fancyvrb} % verbatim replacement that allows latex
    \usepackage{grffile} % extends the file name processing of package graphics
                         % to support a larger range
    \makeatletter % fix for old versions of grffile with XeLaTeX
    \@ifpackagelater{grffile}{2019/11/01}
    {
      % Do nothing on new versions
    }
    {
      \def\Gread@@xetex#1{%
        \IfFileExists{"\Gin@base".bb}%
        {\Gread@eps{\Gin@base.bb}}%
        {\Gread@@xetex@aux#1}%
      }
    }
    \makeatother
    \usepackage[Export]{adjustbox} % Used to constrain images to a maximum size
    \adjustboxset{max size={0.9\linewidth}{0.9\paperheight}}

    % The hyperref package gives us a pdf with properly built
    % internal navigation ('pdf bookmarks' for the table of contents,
    % internal cross-reference links, web links for URLs, etc.)
    \usepackage{hyperref}
    % The default LaTeX title has an obnoxious amount of whitespace. By default,
    % titling removes some of it. It also provides customization options.
    \usepackage{titling}
    \usepackage{longtable} % longtable support required by pandoc >1.10
    \usepackage{booktabs}  % table support for pandoc > 1.12.2
    \usepackage{array}     % table support for pandoc >= 2.11.3
    \usepackage{calc}      % table minipage width calculation for pandoc >= 2.11.1
    \usepackage[inline]{enumitem} % IRkernel/repr support (it uses the enumerate* environment)
    \usepackage[normalem]{ulem} % ulem is needed to support strikethroughs (\sout)
                                % normalem makes italics be italics, not underlines
    \usepackage{soul}      % strikethrough (\st) support for pandoc >= 3.0.0
    \usepackage{mathrsfs}
     % load all other packages
% For cyrillic symbols
\usepackage[english, russian]{babel}


    
    % Colors for the hyperref package
    \definecolor{urlcolor}{rgb}{0,.145,.698}
    \definecolor{linkcolor}{rgb}{.71,0.21,0.01}
    \definecolor{citecolor}{rgb}{.12,.54,.11}

    % ANSI colors
    \definecolor{ansi-black}{HTML}{3E424D}
    \definecolor{ansi-black-intense}{HTML}{282C36}
    \definecolor{ansi-red}{HTML}{E75C58}
    \definecolor{ansi-red-intense}{HTML}{B22B31}
    \definecolor{ansi-green}{HTML}{00A250}
    \definecolor{ansi-green-intense}{HTML}{007427}
    \definecolor{ansi-yellow}{HTML}{DDB62B}
    \definecolor{ansi-yellow-intense}{HTML}{B27D12}
    \definecolor{ansi-blue}{HTML}{208FFB}
    \definecolor{ansi-blue-intense}{HTML}{0065CA}
    \definecolor{ansi-magenta}{HTML}{D160C4}
    \definecolor{ansi-magenta-intense}{HTML}{A03196}
    \definecolor{ansi-cyan}{HTML}{60C6C8}
    \definecolor{ansi-cyan-intense}{HTML}{258F8F}
    \definecolor{ansi-white}{HTML}{C5C1B4}
    \definecolor{ansi-white-intense}{HTML}{A1A6B2}
    \definecolor{ansi-default-inverse-fg}{HTML}{FFFFFF}
    \definecolor{ansi-default-inverse-bg}{HTML}{000000}

    % common color for the border for error outputs.
    \definecolor{outerrorbackground}{HTML}{FFDFDF}

    % commands and environments needed by pandoc snippets
    % extracted from the output of `pandoc -s`
    \providecommand{\tightlist}{%
      \setlength{\itemsep}{0pt}\setlength{\parskip}{0pt}}
    \DefineVerbatimEnvironment{Highlighting}{Verbatim}{commandchars=\\\{\}}
    % Add ',fontsize=\small' for more characters per line
    \newenvironment{Shaded}{}{}
    \newcommand{\KeywordTok}[1]{\textcolor[rgb]{0.00,0.44,0.13}{\textbf{{#1}}}}
    \newcommand{\DataTypeTok}[1]{\textcolor[rgb]{0.56,0.13,0.00}{{#1}}}
    \newcommand{\DecValTok}[1]{\textcolor[rgb]{0.25,0.63,0.44}{{#1}}}
    \newcommand{\BaseNTok}[1]{\textcolor[rgb]{0.25,0.63,0.44}{{#1}}}
    \newcommand{\FloatTok}[1]{\textcolor[rgb]{0.25,0.63,0.44}{{#1}}}
    \newcommand{\CharTok}[1]{\textcolor[rgb]{0.25,0.44,0.63}{{#1}}}
    \newcommand{\StringTok}[1]{\textcolor[rgb]{0.25,0.44,0.63}{{#1}}}
    \newcommand{\CommentTok}[1]{\textcolor[rgb]{0.38,0.63,0.69}{\textit{{#1}}}}
    \newcommand{\OtherTok}[1]{\textcolor[rgb]{0.00,0.44,0.13}{{#1}}}
    \newcommand{\AlertTok}[1]{\textcolor[rgb]{1.00,0.00,0.00}{\textbf{{#1}}}}
    \newcommand{\FunctionTok}[1]{\textcolor[rgb]{0.02,0.16,0.49}{{#1}}}
    \newcommand{\RegionMarkerTok}[1]{{#1}}
    \newcommand{\ErrorTok}[1]{\textcolor[rgb]{1.00,0.00,0.00}{\textbf{{#1}}}}
    \newcommand{\NormalTok}[1]{{#1}}

    % Additional commands for more recent versions of Pandoc
    \newcommand{\ConstantTok}[1]{\textcolor[rgb]{0.53,0.00,0.00}{{#1}}}
    \newcommand{\SpecialCharTok}[1]{\textcolor[rgb]{0.25,0.44,0.63}{{#1}}}
    \newcommand{\VerbatimStringTok}[1]{\textcolor[rgb]{0.25,0.44,0.63}{{#1}}}
    \newcommand{\SpecialStringTok}[1]{\textcolor[rgb]{0.73,0.40,0.53}{{#1}}}
    \newcommand{\ImportTok}[1]{{#1}}
    \newcommand{\DocumentationTok}[1]{\textcolor[rgb]{0.73,0.13,0.13}{\textit{{#1}}}}
    \newcommand{\AnnotationTok}[1]{\textcolor[rgb]{0.38,0.63,0.69}{\textbf{\textit{{#1}}}}}
    \newcommand{\CommentVarTok}[1]{\textcolor[rgb]{0.38,0.63,0.69}{\textbf{\textit{{#1}}}}}
    \newcommand{\VariableTok}[1]{\textcolor[rgb]{0.10,0.09,0.49}{{#1}}}
    \newcommand{\ControlFlowTok}[1]{\textcolor[rgb]{0.00,0.44,0.13}{\textbf{{#1}}}}
    \newcommand{\OperatorTok}[1]{\textcolor[rgb]{0.40,0.40,0.40}{{#1}}}
    \newcommand{\BuiltInTok}[1]{{#1}}
    \newcommand{\ExtensionTok}[1]{{#1}}
    \newcommand{\PreprocessorTok}[1]{\textcolor[rgb]{0.74,0.48,0.00}{{#1}}}
    \newcommand{\AttributeTok}[1]{\textcolor[rgb]{0.49,0.56,0.16}{{#1}}}
    \newcommand{\InformationTok}[1]{\textcolor[rgb]{0.38,0.63,0.69}{\textbf{\textit{{#1}}}}}
    \newcommand{\WarningTok}[1]{\textcolor[rgb]{0.38,0.63,0.69}{\textbf{\textit{{#1}}}}}


    % Define a nice break command that doesn't care if a line doesn't already
    % exist.
    \def\br{\hspace*{\fill} \\* }
    % Math Jax compatibility definitions
    \def\gt{>}
    \def\lt{<}
    \let\Oldtex\TeX
    \let\Oldlatex\LaTeX
    \renewcommand{\TeX}{\textrm{\Oldtex}}
    \renewcommand{\LaTeX}{\textrm{\Oldlatex}}
    % Document parameters
    % Document title
    \title{risk-profile}
    
    
    
    
    
    
    
% Pygments definitions
\makeatletter
\def\PY@reset{\let\PY@it=\relax \let\PY@bf=\relax%
    \let\PY@ul=\relax \let\PY@tc=\relax%
    \let\PY@bc=\relax \let\PY@ff=\relax}
\def\PY@tok#1{\csname PY@tok@#1\endcsname}
\def\PY@toks#1+{\ifx\relax#1\empty\else%
    \PY@tok{#1}\expandafter\PY@toks\fi}
\def\PY@do#1{\PY@bc{\PY@tc{\PY@ul{%
    \PY@it{\PY@bf{\PY@ff{#1}}}}}}}
\def\PY#1#2{\PY@reset\PY@toks#1+\relax+\PY@do{#2}}

\@namedef{PY@tok@w}{\def\PY@tc##1{\textcolor[rgb]{0.73,0.73,0.73}{##1}}}
\@namedef{PY@tok@c}{\let\PY@it=\textit\def\PY@tc##1{\textcolor[rgb]{0.24,0.48,0.48}{##1}}}
\@namedef{PY@tok@cp}{\def\PY@tc##1{\textcolor[rgb]{0.61,0.40,0.00}{##1}}}
\@namedef{PY@tok@k}{\let\PY@bf=\textbf\def\PY@tc##1{\textcolor[rgb]{0.00,0.50,0.00}{##1}}}
\@namedef{PY@tok@kp}{\def\PY@tc##1{\textcolor[rgb]{0.00,0.50,0.00}{##1}}}
\@namedef{PY@tok@kt}{\def\PY@tc##1{\textcolor[rgb]{0.69,0.00,0.25}{##1}}}
\@namedef{PY@tok@o}{\def\PY@tc##1{\textcolor[rgb]{0.40,0.40,0.40}{##1}}}
\@namedef{PY@tok@ow}{\let\PY@bf=\textbf\def\PY@tc##1{\textcolor[rgb]{0.67,0.13,1.00}{##1}}}
\@namedef{PY@tok@nb}{\def\PY@tc##1{\textcolor[rgb]{0.00,0.50,0.00}{##1}}}
\@namedef{PY@tok@nf}{\def\PY@tc##1{\textcolor[rgb]{0.00,0.00,1.00}{##1}}}
\@namedef{PY@tok@nc}{\let\PY@bf=\textbf\def\PY@tc##1{\textcolor[rgb]{0.00,0.00,1.00}{##1}}}
\@namedef{PY@tok@nn}{\let\PY@bf=\textbf\def\PY@tc##1{\textcolor[rgb]{0.00,0.00,1.00}{##1}}}
\@namedef{PY@tok@ne}{\let\PY@bf=\textbf\def\PY@tc##1{\textcolor[rgb]{0.80,0.25,0.22}{##1}}}
\@namedef{PY@tok@nv}{\def\PY@tc##1{\textcolor[rgb]{0.10,0.09,0.49}{##1}}}
\@namedef{PY@tok@no}{\def\PY@tc##1{\textcolor[rgb]{0.53,0.00,0.00}{##1}}}
\@namedef{PY@tok@nl}{\def\PY@tc##1{\textcolor[rgb]{0.46,0.46,0.00}{##1}}}
\@namedef{PY@tok@ni}{\let\PY@bf=\textbf\def\PY@tc##1{\textcolor[rgb]{0.44,0.44,0.44}{##1}}}
\@namedef{PY@tok@na}{\def\PY@tc##1{\textcolor[rgb]{0.41,0.47,0.13}{##1}}}
\@namedef{PY@tok@nt}{\let\PY@bf=\textbf\def\PY@tc##1{\textcolor[rgb]{0.00,0.50,0.00}{##1}}}
\@namedef{PY@tok@nd}{\def\PY@tc##1{\textcolor[rgb]{0.67,0.13,1.00}{##1}}}
\@namedef{PY@tok@s}{\def\PY@tc##1{\textcolor[rgb]{0.73,0.13,0.13}{##1}}}
\@namedef{PY@tok@sd}{\let\PY@it=\textit\def\PY@tc##1{\textcolor[rgb]{0.73,0.13,0.13}{##1}}}
\@namedef{PY@tok@si}{\let\PY@bf=\textbf\def\PY@tc##1{\textcolor[rgb]{0.64,0.35,0.47}{##1}}}
\@namedef{PY@tok@se}{\let\PY@bf=\textbf\def\PY@tc##1{\textcolor[rgb]{0.67,0.36,0.12}{##1}}}
\@namedef{PY@tok@sr}{\def\PY@tc##1{\textcolor[rgb]{0.64,0.35,0.47}{##1}}}
\@namedef{PY@tok@ss}{\def\PY@tc##1{\textcolor[rgb]{0.10,0.09,0.49}{##1}}}
\@namedef{PY@tok@sx}{\def\PY@tc##1{\textcolor[rgb]{0.00,0.50,0.00}{##1}}}
\@namedef{PY@tok@m}{\def\PY@tc##1{\textcolor[rgb]{0.40,0.40,0.40}{##1}}}
\@namedef{PY@tok@gh}{\let\PY@bf=\textbf\def\PY@tc##1{\textcolor[rgb]{0.00,0.00,0.50}{##1}}}
\@namedef{PY@tok@gu}{\let\PY@bf=\textbf\def\PY@tc##1{\textcolor[rgb]{0.50,0.00,0.50}{##1}}}
\@namedef{PY@tok@gd}{\def\PY@tc##1{\textcolor[rgb]{0.63,0.00,0.00}{##1}}}
\@namedef{PY@tok@gi}{\def\PY@tc##1{\textcolor[rgb]{0.00,0.52,0.00}{##1}}}
\@namedef{PY@tok@gr}{\def\PY@tc##1{\textcolor[rgb]{0.89,0.00,0.00}{##1}}}
\@namedef{PY@tok@ge}{\let\PY@it=\textit}
\@namedef{PY@tok@gs}{\let\PY@bf=\textbf}
\@namedef{PY@tok@ges}{\let\PY@bf=\textbf\let\PY@it=\textit}
\@namedef{PY@tok@gp}{\let\PY@bf=\textbf\def\PY@tc##1{\textcolor[rgb]{0.00,0.00,0.50}{##1}}}
\@namedef{PY@tok@go}{\def\PY@tc##1{\textcolor[rgb]{0.44,0.44,0.44}{##1}}}
\@namedef{PY@tok@gt}{\def\PY@tc##1{\textcolor[rgb]{0.00,0.27,0.87}{##1}}}
\@namedef{PY@tok@err}{\def\PY@bc##1{{\setlength{\fboxsep}{\string -\fboxrule}\fcolorbox[rgb]{1.00,0.00,0.00}{1,1,1}{\strut ##1}}}}
\@namedef{PY@tok@kc}{\let\PY@bf=\textbf\def\PY@tc##1{\textcolor[rgb]{0.00,0.50,0.00}{##1}}}
\@namedef{PY@tok@kd}{\let\PY@bf=\textbf\def\PY@tc##1{\textcolor[rgb]{0.00,0.50,0.00}{##1}}}
\@namedef{PY@tok@kn}{\let\PY@bf=\textbf\def\PY@tc##1{\textcolor[rgb]{0.00,0.50,0.00}{##1}}}
\@namedef{PY@tok@kr}{\let\PY@bf=\textbf\def\PY@tc##1{\textcolor[rgb]{0.00,0.50,0.00}{##1}}}
\@namedef{PY@tok@bp}{\def\PY@tc##1{\textcolor[rgb]{0.00,0.50,0.00}{##1}}}
\@namedef{PY@tok@fm}{\def\PY@tc##1{\textcolor[rgb]{0.00,0.00,1.00}{##1}}}
\@namedef{PY@tok@vc}{\def\PY@tc##1{\textcolor[rgb]{0.10,0.09,0.49}{##1}}}
\@namedef{PY@tok@vg}{\def\PY@tc##1{\textcolor[rgb]{0.10,0.09,0.49}{##1}}}
\@namedef{PY@tok@vi}{\def\PY@tc##1{\textcolor[rgb]{0.10,0.09,0.49}{##1}}}
\@namedef{PY@tok@vm}{\def\PY@tc##1{\textcolor[rgb]{0.10,0.09,0.49}{##1}}}
\@namedef{PY@tok@sa}{\def\PY@tc##1{\textcolor[rgb]{0.73,0.13,0.13}{##1}}}
\@namedef{PY@tok@sb}{\def\PY@tc##1{\textcolor[rgb]{0.73,0.13,0.13}{##1}}}
\@namedef{PY@tok@sc}{\def\PY@tc##1{\textcolor[rgb]{0.73,0.13,0.13}{##1}}}
\@namedef{PY@tok@dl}{\def\PY@tc##1{\textcolor[rgb]{0.73,0.13,0.13}{##1}}}
\@namedef{PY@tok@s2}{\def\PY@tc##1{\textcolor[rgb]{0.73,0.13,0.13}{##1}}}
\@namedef{PY@tok@sh}{\def\PY@tc##1{\textcolor[rgb]{0.73,0.13,0.13}{##1}}}
\@namedef{PY@tok@s1}{\def\PY@tc##1{\textcolor[rgb]{0.73,0.13,0.13}{##1}}}
\@namedef{PY@tok@mb}{\def\PY@tc##1{\textcolor[rgb]{0.40,0.40,0.40}{##1}}}
\@namedef{PY@tok@mf}{\def\PY@tc##1{\textcolor[rgb]{0.40,0.40,0.40}{##1}}}
\@namedef{PY@tok@mh}{\def\PY@tc##1{\textcolor[rgb]{0.40,0.40,0.40}{##1}}}
\@namedef{PY@tok@mi}{\def\PY@tc##1{\textcolor[rgb]{0.40,0.40,0.40}{##1}}}
\@namedef{PY@tok@il}{\def\PY@tc##1{\textcolor[rgb]{0.40,0.40,0.40}{##1}}}
\@namedef{PY@tok@mo}{\def\PY@tc##1{\textcolor[rgb]{0.40,0.40,0.40}{##1}}}
\@namedef{PY@tok@ch}{\let\PY@it=\textit\def\PY@tc##1{\textcolor[rgb]{0.24,0.48,0.48}{##1}}}
\@namedef{PY@tok@cm}{\let\PY@it=\textit\def\PY@tc##1{\textcolor[rgb]{0.24,0.48,0.48}{##1}}}
\@namedef{PY@tok@cpf}{\let\PY@it=\textit\def\PY@tc##1{\textcolor[rgb]{0.24,0.48,0.48}{##1}}}
\@namedef{PY@tok@c1}{\let\PY@it=\textit\def\PY@tc##1{\textcolor[rgb]{0.24,0.48,0.48}{##1}}}
\@namedef{PY@tok@cs}{\let\PY@it=\textit\def\PY@tc##1{\textcolor[rgb]{0.24,0.48,0.48}{##1}}}

\def\PYZbs{\char`\\}
\def\PYZus{\char`\_}
\def\PYZob{\char`\{}
\def\PYZcb{\char`\}}
\def\PYZca{\char`\^}
\def\PYZam{\char`\&}
\def\PYZlt{\char`\<}
\def\PYZgt{\char`\>}
\def\PYZsh{\char`\#}
\def\PYZpc{\char`\%}
\def\PYZdl{\char`\$}
\def\PYZhy{\char`\-}
\def\PYZsq{\char`\'}
\def\PYZdq{\char`\"}
\def\PYZti{\char`\~}
% for compatibility with earlier versions
\def\PYZat{@}
\def\PYZlb{[}
\def\PYZrb{]}
\makeatother


    % For linebreaks inside Verbatim environment from package fancyvrb.
    \makeatletter
        \newbox\Wrappedcontinuationbox
        \newbox\Wrappedvisiblespacebox
        \newcommand*\Wrappedvisiblespace {\textcolor{red}{\textvisiblespace}}
        \newcommand*\Wrappedcontinuationsymbol {\textcolor{red}{\llap{\tiny$\m@th\hookrightarrow$}}}
        \newcommand*\Wrappedcontinuationindent {3ex }
        \newcommand*\Wrappedafterbreak {\kern\Wrappedcontinuationindent\copy\Wrappedcontinuationbox}
        % Take advantage of the already applied Pygments mark-up to insert
        % potential linebreaks for TeX processing.
        %        {, <, #, %, $, ' and ": go to next line.
        %        _, }, ^, &, >, - and ~: stay at end of broken line.
        % Use of \textquotesingle for straight quote.
        \newcommand*\Wrappedbreaksatspecials {%
            \def\PYGZus{\discretionary{\char`\_}{\Wrappedafterbreak}{\char`\_}}%
            \def\PYGZob{\discretionary{}{\Wrappedafterbreak\char`\{}{\char`\{}}%
            \def\PYGZcb{\discretionary{\char`\}}{\Wrappedafterbreak}{\char`\}}}%
            \def\PYGZca{\discretionary{\char`\^}{\Wrappedafterbreak}{\char`\^}}%
            \def\PYGZam{\discretionary{\char`\&}{\Wrappedafterbreak}{\char`\&}}%
            \def\PYGZlt{\discretionary{}{\Wrappedafterbreak\char`\<}{\char`\<}}%
            \def\PYGZgt{\discretionary{\char`\>}{\Wrappedafterbreak}{\char`\>}}%
            \def\PYGZsh{\discretionary{}{\Wrappedafterbreak\char`\#}{\char`\#}}%
            \def\PYGZpc{\discretionary{}{\Wrappedafterbreak\char`\%}{\char`\%}}%
            \def\PYGZdl{\discretionary{}{\Wrappedafterbreak\char`\$}{\char`\$}}%
            \def\PYGZhy{\discretionary{\char`\-}{\Wrappedafterbreak}{\char`\-}}%
            \def\PYGZsq{\discretionary{}{\Wrappedafterbreak\textquotesingle}{\textquotesingle}}%
            \def\PYGZdq{\discretionary{}{\Wrappedafterbreak\char`\"}{\char`\"}}%
            \def\PYGZti{\discretionary{\char`\~}{\Wrappedafterbreak}{\char`\~}}%
        }
        % Some characters . , ; ? ! / are not pygmentized.
        % This macro makes them "active" and they will insert potential linebreaks
        \newcommand*\Wrappedbreaksatpunct {%
            \lccode`\~`\.\lowercase{\def~}{\discretionary{\hbox{\char`\.}}{\Wrappedafterbreak}{\hbox{\char`\.}}}%
            \lccode`\~`\,\lowercase{\def~}{\discretionary{\hbox{\char`\,}}{\Wrappedafterbreak}{\hbox{\char`\,}}}%
            \lccode`\~`\;\lowercase{\def~}{\discretionary{\hbox{\char`\;}}{\Wrappedafterbreak}{\hbox{\char`\;}}}%
            \lccode`\~`\:\lowercase{\def~}{\discretionary{\hbox{\char`\:}}{\Wrappedafterbreak}{\hbox{\char`\:}}}%
            \lccode`\~`\?\lowercase{\def~}{\discretionary{\hbox{\char`\?}}{\Wrappedafterbreak}{\hbox{\char`\?}}}%
            \lccode`\~`\!\lowercase{\def~}{\discretionary{\hbox{\char`\!}}{\Wrappedafterbreak}{\hbox{\char`\!}}}%
            \lccode`\~`\/\lowercase{\def~}{\discretionary{\hbox{\char`\/}}{\Wrappedafterbreak}{\hbox{\char`\/}}}%
            \catcode`\.\active
            \catcode`\,\active
            \catcode`\;\active
            \catcode`\:\active
            \catcode`\?\active
            \catcode`\!\active
            \catcode`\/\active
            \lccode`\~`\~
        }
    \makeatother

    \let\OriginalVerbatim=\Verbatim
    \makeatletter
    \renewcommand{\Verbatim}[1][1]{%
        %\parskip\z@skip
        \sbox\Wrappedcontinuationbox {\Wrappedcontinuationsymbol}%
        \sbox\Wrappedvisiblespacebox {\FV@SetupFont\Wrappedvisiblespace}%
        \def\FancyVerbFormatLine ##1{\hsize\linewidth
            \vtop{\raggedright\hyphenpenalty\z@\exhyphenpenalty\z@
                \doublehyphendemerits\z@\finalhyphendemerits\z@
                \strut ##1\strut}%
        }%
        % If the linebreak is at a space, the latter will be displayed as visible
        % space at end of first line, and a continuation symbol starts next line.
        % Stretch/shrink are however usually zero for typewriter font.
        \def\FV@Space {%
            \nobreak\hskip\z@ plus\fontdimen3\font minus\fontdimen4\font
            \discretionary{\copy\Wrappedvisiblespacebox}{\Wrappedafterbreak}
            {\kern\fontdimen2\font}%
        }%

        % Allow breaks at special characters using \PYG... macros.
        \Wrappedbreaksatspecials
        % Breaks at punctuation characters . , ; ? ! and / need catcode=\active
        \OriginalVerbatim[#1,codes*=\Wrappedbreaksatpunct]%
    }
    \makeatother

    % Exact colors from NB
    \definecolor{incolor}{HTML}{303F9F}
    \definecolor{outcolor}{HTML}{D84315}
    \definecolor{cellborder}{HTML}{CFCFCF}
    \definecolor{cellbackground}{HTML}{F7F7F7}

    % prompt
    \makeatletter
    \newcommand{\boxspacing}{\kern\kvtcb@left@rule\kern\kvtcb@boxsep}
    \makeatother
    \newcommand{\prompt}[4]{
        {\ttfamily\llap{{\color{#2}[#3]:\hspace{3pt}#4}}\vspace{-\baselineskip}}
    }
    

    
    % Prevent overflowing lines due to hard-to-break entities
    \sloppy
    % Setup hyperref package
    \hypersetup{
      breaklinks=true,  % so long urls are correctly broken across lines
      colorlinks=true,
      urlcolor=urlcolor,
      linkcolor=linkcolor,
      citecolor=citecolor,
      }
    % Slightly bigger margins than the latex defaults
    
    \geometry{verbose,tmargin=1in,bmargin=1in,lmargin=1in,rmargin=1in}
    
    

\begin{document}
    
    \maketitle
    
    

    
    \section{Определение инвестиционного профиля клиента на основе
алгоритмов нечеткой
логики}\label{ux43eux43fux440ux435ux434ux435ux43bux435ux43dux438ux435-ux438ux43dux432ux435ux441ux442ux438ux446ux438ux43eux43dux43dux43eux433ux43e-ux43fux440ux43eux444ux438ux43bux44f-ux43aux43bux438ux435ux43dux442ux430-ux43dux430-ux43eux441ux43dux43eux432ux435-ux430ux43bux433ux43eux440ux438ux442ux43cux43eux432-ux43dux435ux447ux435ux442ux43aux43eux439-ux43bux43eux433ux438ux43aux438}

Сегодня управляющие компании предлагают инвесторам, работающим на
фондовом рынке, многочисленные стратегии управление активами. Однако
далеко не каждому клиенту подойдет типичная стратегия, разработанная
управляющим. Одних интересует максимальная доходность, других --
положительный результат при приемлемом уровне рисков. Определить
оптимальный вариант вложений помогает инвестиционный профиль.

\subsection{Понятие инвестиционного
профиля}\label{ux43fux43eux43dux44fux442ux438ux435-ux438ux43dux432ux435ux441ux442ux438ux446ux438ux43eux43dux43dux43eux433ux43e-ux43fux440ux43eux444ux438ux43bux44f}

Инвестиционный профиль (риск-профиль) -- характеристика инвестора,
описывающий поведение на финансовом рынке. Составление такого профиля
помогает определить запросы клиента, его склонность к рискам и
готовность их принимать.

В зависимости от типа инвестиционного профиля можно:

\begin{itemize}
\tightlist
\item
  подобрать эффективную стратегию инвестирования
\item
  собрать оптимальный портфель ценных бумаг и других активов
\item
  разработать механизмы реакции на различные рыночные события
\end{itemize}

Инвестиционный профиль Клиента может быть определен посредством
заполнения онлайн-анкеты. Целью анкетирования является получение
сведений о клиенте, позволяющих выявить:

\begin{itemize}
\tightlist
\item
  инвестиционные цели клиента, включая ожидаемую доходность
\item
  допустимый риск (риск, который способен нести клиент)
\item
  инвестиционный горизонт
\end{itemize}

\section{Постановка
задачи}\label{ux43fux43eux441ux442ux430ux43dux43eux432ux43aux430-ux437ux430ux434ux430ux447ux438}

Основной целью создания программы является автоматизации процедуры
установления инвестиционного профиля клиента на основе нечеткой логики.

В соответствии с поставленной целью, создаваемая система должна уметь
решать следующие задачи:

\begin{itemize}
\tightlist
\item
  считывание значений параметров по которым определяется инвестиционный
  профиль
\item
  расчет инвестиционного профиля на основе нечеткой логики
\item
  вывод значений параметров полученного инвестиционного профиля
\end{itemize}

\subsection{Входные
данные}\label{ux432ux445ux43eux434ux43dux44bux435-ux434ux430ux43dux43dux44bux435}

\subsubsection{1. Time horizon (горизонт
инвестирования)}\label{time-horizon-ux433ux43eux440ux438ux437ux43eux43dux442-ux438ux43dux432ux435ux441ux442ux438ux440ux43eux432ux430ux43dux438ux44f}

Определяется как суммарный скоринг ответов на вопросы онлайн-анкеты:

\paragraph{1.1 Срок
инвестирования}\label{ux441ux440ux43eux43a-ux438ux43dux432ux435ux441ux442ux438ux440ux43eux432ux430ux43dux438ux44f}

\begin{itemize}
\tightlist
\item[$\square$]
  менее одного года (1 балл)
\item[$\square$]
  3-5 лет (3 балла)
\item[$\square$]
  6-10 лет (7 баллов)
\item[$\square$]
  11 лет и более (10 баллов)
\end{itemize}

\paragraph{1.2 Срок начала использования инвестиционных
накоплений}\label{ux441ux440ux43eux43a-ux43dux430ux447ux430ux43bux430-ux438ux441ux43fux43eux43bux44cux437ux43eux432ux430ux43dux438ux44f-ux438ux43dux432ux435ux441ux442ux438ux446ux438ux43eux43dux43dux44bux445-ux43dux430ux43aux43eux43fux43bux435ux43dux438ux439}

\begin{itemize}
\tightlist
\item[$\square$]
  менее чем через 2 года (0 баллов)
\item[$\square$]
  2-5 лет (1 балл)
\item[$\square$]
  6-10 лет (4 балла)
\item[$\square$]
  11 лет и более (8 баллов)
\end{itemize}

\paragraph{Лингвистическая переменная - Time horizon
score}\label{ux43bux438ux43dux433ux432ux438ux441ux442ux438ux447ux435ux441ux43aux430ux44f-ux43fux435ux440ux435ux43cux435ux43dux43dux430ux44f---time-horizon-score}

Определяется по сумме набранных баллов {[}1 - 18{]} при ответе на
вопросы анкеты 1.1 и 1.2

\begin{itemize}
\tightlist
\item
  lower (ML)
\item
  low (LL)
\item
  average (ZE)
\item
  high (HH)
\item
  higher (MH)
\end{itemize}

\subsubsection{2. Толерантность к
риску}\label{ux442ux43eux43bux435ux440ux430ux43dux442ux43dux43eux441ux442ux44c-ux43a-ux440ux438ux441ux43aux443}

Определяется как суммарный скоринг ответов на вопросы:

\paragraph{2.1 Уровень знаний на финансовом
рынке}\label{ux443ux440ux43eux432ux435ux43dux44c-ux437ux43dux430ux43dux438ux439-ux43dux430-ux444ux438ux43dux430ux43dux441ux43eux432ux43eux43c-ux440ux44bux43dux43aux435}

\begin{itemize}
\tightlist
\item[$\square$]
  нет знаний (0 баллов)
\item[$\square$]
  есть базовые знания (3 балла)
\item[$\square$]
  опытный инвестор (7 баллов)
\item[$\square$]
  эксперт в инвестициях (10 баллов)
\end{itemize}

\paragraph{2.2 Цель
инвестиций}\label{ux446ux435ux43bux44c-ux438ux43dux432ux435ux441ux442ux438ux446ux438ux439}

\begin{itemize}
\tightlist
\item[$\square$]
  сохранение капитала и накоплений (1 балл)
\item[$\square$]
  пенсионные накопления (обеспеченное будущее) (5 баллов)
\item[$\square$]
  накопление для определенных целей (крупных покупок) (7 баллов)
\item[$\square$]
  получение максимального дохода (10 баллов)
\end{itemize}

\paragraph{2.3 Ожидаемая доходность от
инвестиций}\label{ux43eux436ux438ux434ux430ux435ux43cux430ux44f-ux434ux43eux445ux43eux434ux43dux43eux441ux442ux44c-ux43eux442-ux438ux43dux432ux435ux441ux442ux438ux446ux438ux439}

\begin{itemize}
\tightlist
\item[$\square$]
  хотя бы немного выше депозита (1 балл)
\item[$\square$]
  около 10\% в год (3 балла)
\item[$\square$]
  около 20\% в год (7 баллов)
\item[$\square$]
  выше 20\% в год (10 баллов)
\end{itemize}

\paragraph{2.4 Отношение к рискам
инвестирования}\label{ux43eux442ux43dux43eux448ux435ux43dux438ux435-ux43a-ux440ux438ux441ux43aux430ux43c-ux438ux43dux432ux435ux441ux442ux438ux440ux43eux432ux430ux43dux438ux44f}

\begin{itemize}
\tightlist
\item[$\square$]
  главное сохранить сбережения, доходность не так важна (0 баллов)
\item[$\square$]
  хотелось бы иметь доходность немного выше выше, чем по банковским
  вкладам (3 балла)
\item[$\square$]
  хотелось бы иметь доходность, превыщающую банковский вклад в 2 раза (7
  баллов)
\item[$\square$]
  готов инвестировать в инструменты, приносящие максимальный доход (10
  баллов)
\end{itemize}

\paragraph{Лингвистическая переменная - Risk tolerance
score}\label{ux43bux438ux43dux433ux432ux438ux441ux442ux438ux447ux435ux441ux43aux430ux44f-ux43fux435ux440ux435ux43cux435ux43dux43dux430ux44f---risk-tolerance-score}

Определяется по сумме набранных баллов {[}0 - 40{]} при ответе на
вопросы онлайн-анкеты 2.1 - 2.4

\begin{itemize}
\tightlist
\item
  lowest (BL)
\item
  lower (ML)
\item
  low (LL)
\item
  average (ZE)
\item
  high (HH)
\item
  higher (MH)
\item
  highest (BH)
\end{itemize}

\subsection{Выходные
данные}\label{ux432ux44bux445ux43eux434ux43dux44bux435-ux434ux430ux43dux43dux44bux435}

\subsubsection{Уровень инвестиционного
риска}\label{ux443ux440ux43eux432ux435ux43dux44c-ux438ux43dux432ux435ux441ux442ux438ux446ux438ux43eux43dux43dux43eux433ux43e-ux440ux438ux441ux43aux430}

Число по шкале от 0-100

\subsubsection{Инвестиционный профиль клиента - Risk
Profile}\label{ux438ux43dux432ux435ux441ux442ux438ux446ux438ux43eux43dux43dux44bux439-ux43fux440ux43eux444ux438ux43bux44c-ux43aux43bux438ux435ux43dux442ux430---risk-profile}

Нечеткое множество:

\begin{itemize}
\tightlist
\item
  Conservative (BC) - консервативный
\item
  Moderately-Conservative (MC) - умеренно-консервативный
\item
  Moderate (MM) - умеренный
\item
  Moderately-Aggressive (MA) - умеренно-агрессивный
\item
  Aggressive (BA) - агрессивный
\end{itemize}

\subsection{Правила}\label{ux43fux440ux430ux432ux438ux43bux430}

Правила заданы матрицей, которая построена по следующим правилам

\begin{itemize}
\tightlist
\item
  строки матрицы - значения лингвистической переменной Time horizon
  score
\item
  столбцы матрицы - значения лингвистической переменной Risk tolerance
  score
\item
  ячейка матрицы - значение выходного параметра - Risk Profile
\end{itemize}

\begin{longtable}[]{@{}llllllll@{}}
\toprule\noalign{}
& BL & ML & LL & ZE & HH & MH & BH \\
\midrule\noalign{}
\endhead
\bottomrule\noalign{}
\endlastfoot
ML & BC & BC & MC & MC & MM & MM & MM \\
LL & BC & BC & MC & MM & MM & MA & MA \\
ZE & BC & MC & MC & MM & MA & MA & BA \\
HH & MC & MC & MM & MM & MA & BA & BA \\
MH & MC & MC & MM & MA & MA & BA & BA \\
\end{longtable}

    \begin{tcolorbox}[breakable, size=fbox, boxrule=1pt, pad at break*=1mm,colback=cellbackground, colframe=cellborder]
\prompt{In}{incolor}{8}{\boxspacing}
\begin{Verbatim}[commandchars=\\\{\}]
\PY{k+kn}{import} \PY{n+nn}{numpy} \PY{k}{as} \PY{n+nn}{np}
\PY{k+kn}{from} \PY{n+nn}{skfuzzy} \PY{k+kn}{import} \PY{n}{control} \PY{k}{as} \PY{n}{ctrl}
\end{Verbatim}
\end{tcolorbox}

    \begin{tcolorbox}[breakable, size=fbox, boxrule=1pt, pad at break*=1mm,colback=cellbackground, colframe=cellborder]
\prompt{In}{incolor}{9}{\boxspacing}
\begin{Verbatim}[commandchars=\\\{\}]
\PY{c+c1}{\PYZsh{} Определяем лингвистические переменные}

\PY{n}{time\PYZus{}horizon\PYZus{}score} \PY{o}{=} \PY{n}{ctrl}\PY{o}{.}\PY{n}{Antecedent}\PY{p}{(}\PY{n}{np}\PY{o}{.}\PY{n}{arange}\PY{p}{(}\PY{l+m+mi}{1}\PY{p}{,} \PY{l+m+mi}{18}\PY{p}{,} \PY{l+m+mi}{1}\PY{p}{)}\PY{p}{,} \PY{l+s+s1}{\PYZsq{}}\PY{l+s+s1}{time horizon score}\PY{l+s+s1}{\PYZsq{}}\PY{p}{)}
\PY{n}{risk\PYZus{}tolerance\PYZus{}score} \PY{o}{=} \PY{n}{ctrl}\PY{o}{.}\PY{n}{Antecedent}\PY{p}{(}\PY{n}{np}\PY{o}{.}\PY{n}{arange}\PY{p}{(}\PY{l+m+mi}{0}\PY{p}{,} \PY{l+m+mi}{40}\PY{p}{,} \PY{l+m+mi}{1}\PY{p}{)}\PY{p}{,} \PY{l+s+s1}{\PYZsq{}}\PY{l+s+s1}{risk tolerance score}\PY{l+s+s1}{\PYZsq{}}\PY{p}{)}
\PY{n}{risk\PYZus{}profile} \PY{o}{=} \PY{n}{ctrl}\PY{o}{.}\PY{n}{Consequent}\PY{p}{(}\PY{n}{np}\PY{o}{.}\PY{n}{arange}\PY{p}{(}\PY{l+m+mi}{1}\PY{p}{,} \PY{l+m+mi}{100}\PY{p}{,} \PY{l+m+mi}{1}\PY{p}{)}\PY{p}{,} \PY{l+s+s1}{\PYZsq{}}\PY{l+s+s1}{risk profile}\PY{l+s+s1}{\PYZsq{}}\PY{p}{)}

\PY{c+c1}{\PYZsh{} Количество состояний переменной}
\PY{n}{time\PYZus{}horizon\PYZus{}score}\PY{o}{.}\PY{n}{automf}\PY{p}{(}\PY{l+m+mi}{5}\PY{p}{,} \PY{n}{variable\PYZus{}type}\PY{o}{=}\PY{l+s+s1}{\PYZsq{}}\PY{l+s+s1}{quant}\PY{l+s+s1}{\PYZsq{}}\PY{p}{)}
\PY{n}{risk\PYZus{}tolerance\PYZus{}score}\PY{o}{.}\PY{n}{automf}\PY{p}{(}\PY{l+m+mi}{7}\PY{p}{,} \PY{n}{variable\PYZus{}type}\PY{o}{=}\PY{l+s+s1}{\PYZsq{}}\PY{l+s+s1}{quant}\PY{l+s+s1}{\PYZsq{}}\PY{p}{)}
\PY{n}{risk\PYZus{}profile}\PY{o}{.}\PY{n}{automf}\PY{p}{(}\PY{l+m+mi}{5}\PY{p}{,} \PY{n}{variable\PYZus{}type}\PY{o}{=}\PY{l+s+s1}{\PYZsq{}}\PY{l+s+s1}{quant}\PY{l+s+s1}{\PYZsq{}}\PY{p}{,} \PY{n}{names}\PY{o}{=}\PY{p}{[}\PY{l+s+s1}{\PYZsq{}}\PY{l+s+s1}{conservative}\PY{l+s+s1}{\PYZsq{}}\PY{p}{,} \PY{l+s+s1}{\PYZsq{}}\PY{l+s+s1}{moderately\PYZhy{}conservative}\PY{l+s+s1}{\PYZsq{}}\PY{p}{,} \PY{l+s+s1}{\PYZsq{}}\PY{l+s+s1}{moderate}\PY{l+s+s1}{\PYZsq{}}\PY{p}{,} \PY{l+s+s1}{\PYZsq{}}\PY{l+s+s1}{moderately\PYZhy{}aggressive}\PY{l+s+s1}{\PYZsq{}}\PY{p}{,} \PY{l+s+s1}{\PYZsq{}}\PY{l+s+s1}{aggressive}\PY{l+s+s1}{\PYZsq{}}\PY{p}{]}\PY{p}{)}

\PY{c+c1}{\PYZsh{} Вывод на экран графиков функции принадлежности}
\PY{n}{time\PYZus{}horizon\PYZus{}score}\PY{o}{.}\PY{n}{view}\PY{p}{(}\PY{p}{)}
\PY{n}{risk\PYZus{}tolerance\PYZus{}score}\PY{o}{.}\PY{n}{view}\PY{p}{(}\PY{p}{)}
\PY{n}{risk\PYZus{}profile}\PY{o}{.}\PY{n}{view}\PY{p}{(}\PY{p}{)}
\end{Verbatim}
\end{tcolorbox}

    \begin{center}
    \adjustimage{max size={0.9\linewidth}{0.9\paperheight}}{output_2_0.png}
    \end{center}
    { \hspace*{\fill} \\}
    
    \begin{center}
    \adjustimage{max size={0.9\linewidth}{0.9\paperheight}}{output_2_1.png}
    \end{center}
    { \hspace*{\fill} \\}
    
    \begin{center}
    \adjustimage{max size={0.9\linewidth}{0.9\paperheight}}{output_2_2.png}
    \end{center}
    { \hspace*{\fill} \\}
    
    \begin{tcolorbox}[breakable, size=fbox, boxrule=1pt, pad at break*=1mm,colback=cellbackground, colframe=cellborder]
\prompt{In}{incolor}{10}{\boxspacing}
\begin{Verbatim}[commandchars=\\\{\}]
\PY{c+c1}{\PYZsh{} Правила}

\PY{n}{th\PYZus{}ml} \PY{o}{=} \PY{n}{time\PYZus{}horizon\PYZus{}score}\PY{p}{[}\PY{l+s+s1}{\PYZsq{}}\PY{l+s+s1}{lower}\PY{l+s+s1}{\PYZsq{}}\PY{p}{]}
\PY{n}{th\PYZus{}ll} \PY{o}{=} \PY{n}{time\PYZus{}horizon\PYZus{}score}\PY{p}{[}\PY{l+s+s1}{\PYZsq{}}\PY{l+s+s1}{low}\PY{l+s+s1}{\PYZsq{}}\PY{p}{]}
\PY{n}{th\PYZus{}ze} \PY{o}{=} \PY{n}{time\PYZus{}horizon\PYZus{}score}\PY{p}{[}\PY{l+s+s1}{\PYZsq{}}\PY{l+s+s1}{average}\PY{l+s+s1}{\PYZsq{}}\PY{p}{]}
\PY{n}{th\PYZus{}hh} \PY{o}{=} \PY{n}{time\PYZus{}horizon\PYZus{}score}\PY{p}{[}\PY{l+s+s1}{\PYZsq{}}\PY{l+s+s1}{high}\PY{l+s+s1}{\PYZsq{}}\PY{p}{]}
\PY{n}{th\PYZus{}mh} \PY{o}{=} \PY{n}{time\PYZus{}horizon\PYZus{}score}\PY{p}{[}\PY{l+s+s1}{\PYZsq{}}\PY{l+s+s1}{higher}\PY{l+s+s1}{\PYZsq{}}\PY{p}{]}

\PY{n}{rt\PYZus{}bl} \PY{o}{=} \PY{n}{risk\PYZus{}tolerance\PYZus{}score}\PY{p}{[}\PY{l+s+s1}{\PYZsq{}}\PY{l+s+s1}{lowest}\PY{l+s+s1}{\PYZsq{}}\PY{p}{]}
\PY{n}{rt\PYZus{}ml} \PY{o}{=} \PY{n}{risk\PYZus{}tolerance\PYZus{}score}\PY{p}{[}\PY{l+s+s1}{\PYZsq{}}\PY{l+s+s1}{lower}\PY{l+s+s1}{\PYZsq{}}\PY{p}{]}
\PY{n}{rt\PYZus{}ll} \PY{o}{=} \PY{n}{risk\PYZus{}tolerance\PYZus{}score}\PY{p}{[}\PY{l+s+s1}{\PYZsq{}}\PY{l+s+s1}{low}\PY{l+s+s1}{\PYZsq{}}\PY{p}{]}
\PY{n}{rt\PYZus{}ze} \PY{o}{=} \PY{n}{risk\PYZus{}tolerance\PYZus{}score}\PY{p}{[}\PY{l+s+s1}{\PYZsq{}}\PY{l+s+s1}{average}\PY{l+s+s1}{\PYZsq{}}\PY{p}{]}
\PY{n}{rt\PYZus{}hh} \PY{o}{=} \PY{n}{risk\PYZus{}tolerance\PYZus{}score}\PY{p}{[}\PY{l+s+s1}{\PYZsq{}}\PY{l+s+s1}{high}\PY{l+s+s1}{\PYZsq{}}\PY{p}{]}
\PY{n}{rt\PYZus{}mh} \PY{o}{=} \PY{n}{risk\PYZus{}tolerance\PYZus{}score}\PY{p}{[}\PY{l+s+s1}{\PYZsq{}}\PY{l+s+s1}{higher}\PY{l+s+s1}{\PYZsq{}}\PY{p}{]}
\PY{n}{rt\PYZus{}bh} \PY{o}{=} \PY{n}{risk\PYZus{}tolerance\PYZus{}score}\PY{p}{[}\PY{l+s+s1}{\PYZsq{}}\PY{l+s+s1}{highest}\PY{l+s+s1}{\PYZsq{}}\PY{p}{]}

\PY{n}{rp\PYZus{}bc} \PY{o}{=} \PY{n}{risk\PYZus{}profile}\PY{p}{[}\PY{l+s+s1}{\PYZsq{}}\PY{l+s+s1}{conservative}\PY{l+s+s1}{\PYZsq{}}\PY{p}{]}
\PY{n}{rp\PYZus{}mc} \PY{o}{=} \PY{n}{risk\PYZus{}profile}\PY{p}{[}\PY{l+s+s1}{\PYZsq{}}\PY{l+s+s1}{moderately\PYZhy{}conservative}\PY{l+s+s1}{\PYZsq{}}\PY{p}{]}
\PY{n}{rp\PYZus{}mm} \PY{o}{=} \PY{n}{risk\PYZus{}profile}\PY{p}{[}\PY{l+s+s1}{\PYZsq{}}\PY{l+s+s1}{moderate}\PY{l+s+s1}{\PYZsq{}}\PY{p}{]}
\PY{n}{rp\PYZus{}ma} \PY{o}{=} \PY{n}{risk\PYZus{}profile}\PY{p}{[}\PY{l+s+s1}{\PYZsq{}}\PY{l+s+s1}{moderately\PYZhy{}aggressive}\PY{l+s+s1}{\PYZsq{}}\PY{p}{]}
\PY{n}{rp\PYZus{}ba} \PY{o}{=} \PY{n}{risk\PYZus{}profile}\PY{p}{[}\PY{l+s+s1}{\PYZsq{}}\PY{l+s+s1}{aggressive}\PY{l+s+s1}{\PYZsq{}}\PY{p}{]}

\PY{c+c1}{\PYZsh{} row 1}
\PY{n}{rule1\PYZus{}1} \PY{o}{=} \PY{n}{ctrl}\PY{o}{.}\PY{n}{Rule}\PY{p}{(}\PY{n}{th\PYZus{}ml} \PY{o}{\PYZam{}} \PY{p}{(}\PY{n}{rt\PYZus{}bl} \PY{o}{|} \PY{n}{rt\PYZus{}ml}\PY{p}{)}\PY{p}{,} \PY{n}{rp\PYZus{}bc}\PY{p}{)}
\PY{n}{rule1\PYZus{}2} \PY{o}{=} \PY{n}{ctrl}\PY{o}{.}\PY{n}{Rule}\PY{p}{(}\PY{n}{th\PYZus{}ml} \PY{o}{\PYZam{}} \PY{p}{(}\PY{n}{rt\PYZus{}ll} \PY{o}{|} \PY{n}{rt\PYZus{}ze}\PY{p}{)}\PY{p}{,} \PY{n}{rp\PYZus{}mc}\PY{p}{)}
\PY{n}{rule1\PYZus{}3} \PY{o}{=} \PY{n}{ctrl}\PY{o}{.}\PY{n}{Rule}\PY{p}{(}\PY{n}{th\PYZus{}ml} \PY{o}{\PYZam{}} \PY{p}{(}\PY{n}{rt\PYZus{}hh} \PY{o}{|} \PY{n}{rt\PYZus{}mh} \PY{o}{|} \PY{n}{rt\PYZus{}bh}\PY{p}{)}\PY{p}{,} \PY{n}{rp\PYZus{}mm}\PY{p}{)}

\PY{c+c1}{\PYZsh{} row 2}
\PY{n}{rule2\PYZus{}1} \PY{o}{=} \PY{n}{ctrl}\PY{o}{.}\PY{n}{Rule}\PY{p}{(}\PY{n}{th\PYZus{}ll} \PY{o}{\PYZam{}} \PY{p}{(}\PY{n}{rt\PYZus{}bl} \PY{o}{|} \PY{n}{rt\PYZus{}ml}\PY{p}{)}\PY{p}{,} \PY{n}{rp\PYZus{}bc}\PY{p}{)}
\PY{n}{rule2\PYZus{}2} \PY{o}{=} \PY{n}{ctrl}\PY{o}{.}\PY{n}{Rule}\PY{p}{(}\PY{n}{th\PYZus{}ll} \PY{o}{\PYZam{}} \PY{n}{rt\PYZus{}ll}\PY{p}{,} \PY{n}{rp\PYZus{}mc}\PY{p}{)}
\PY{n}{rule2\PYZus{}3} \PY{o}{=} \PY{n}{ctrl}\PY{o}{.}\PY{n}{Rule}\PY{p}{(}\PY{n}{th\PYZus{}ll} \PY{o}{\PYZam{}} \PY{p}{(}\PY{n}{rt\PYZus{}ze} \PY{o}{|} \PY{n}{rt\PYZus{}hh}\PY{p}{)}\PY{p}{,} \PY{n}{rp\PYZus{}mm}\PY{p}{)}
\PY{n}{rule2\PYZus{}4} \PY{o}{=} \PY{n}{ctrl}\PY{o}{.}\PY{n}{Rule}\PY{p}{(}\PY{n}{th\PYZus{}ll} \PY{o}{\PYZam{}} \PY{p}{(}\PY{n}{rt\PYZus{}mh} \PY{o}{|} \PY{n}{rt\PYZus{}bh}\PY{p}{)}\PY{p}{,} \PY{n}{rp\PYZus{}ma}\PY{p}{)}

\PY{c+c1}{\PYZsh{} row 3}
\PY{n}{rule3\PYZus{}1} \PY{o}{=} \PY{n}{ctrl}\PY{o}{.}\PY{n}{Rule}\PY{p}{(}\PY{n}{th\PYZus{}ze} \PY{o}{\PYZam{}} \PY{n}{rt\PYZus{}bl}\PY{p}{,} \PY{n}{rp\PYZus{}bc}\PY{p}{)}
\PY{n}{rule3\PYZus{}2} \PY{o}{=} \PY{n}{ctrl}\PY{o}{.}\PY{n}{Rule}\PY{p}{(}\PY{n}{th\PYZus{}ze} \PY{o}{\PYZam{}} \PY{p}{(}\PY{n}{rt\PYZus{}ml} \PY{o}{|} \PY{n}{rt\PYZus{}ll}\PY{p}{)}\PY{p}{,} \PY{n}{rp\PYZus{}mc}\PY{p}{)}
\PY{n}{rule3\PYZus{}3} \PY{o}{=} \PY{n}{ctrl}\PY{o}{.}\PY{n}{Rule}\PY{p}{(}\PY{n}{th\PYZus{}ze} \PY{o}{\PYZam{}} \PY{n}{rt\PYZus{}ze}\PY{p}{,} \PY{n}{rp\PYZus{}mm}\PY{p}{)}
\PY{n}{rule3\PYZus{}4} \PY{o}{=} \PY{n}{ctrl}\PY{o}{.}\PY{n}{Rule}\PY{p}{(}\PY{n}{th\PYZus{}ze} \PY{o}{\PYZam{}} \PY{p}{(}\PY{n}{rt\PYZus{}hh} \PY{o}{|} \PY{n}{rt\PYZus{}mh}\PY{p}{)}\PY{p}{,} \PY{n}{rp\PYZus{}ma}\PY{p}{)}
\PY{n}{rule3\PYZus{}5} \PY{o}{=} \PY{n}{ctrl}\PY{o}{.}\PY{n}{Rule}\PY{p}{(}\PY{n}{th\PYZus{}ze} \PY{o}{\PYZam{}} \PY{n}{rt\PYZus{}bh}\PY{p}{,} \PY{n}{rp\PYZus{}ba}\PY{p}{)}

\PY{c+c1}{\PYZsh{} row 4}
\PY{n}{rule4\PYZus{}1} \PY{o}{=} \PY{n}{ctrl}\PY{o}{.}\PY{n}{Rule}\PY{p}{(}\PY{n}{th\PYZus{}hh} \PY{o}{\PYZam{}} \PY{p}{(}\PY{n}{rt\PYZus{}bl} \PY{o}{|} \PY{n}{rt\PYZus{}ml}\PY{p}{)}\PY{p}{,} \PY{n}{rp\PYZus{}mc}\PY{p}{)}
\PY{n}{rule4\PYZus{}2} \PY{o}{=} \PY{n}{ctrl}\PY{o}{.}\PY{n}{Rule}\PY{p}{(}\PY{n}{th\PYZus{}hh} \PY{o}{\PYZam{}} \PY{p}{(}\PY{n}{rt\PYZus{}ll} \PY{o}{|} \PY{n}{rt\PYZus{}ze}\PY{p}{)}\PY{p}{,} \PY{n}{rp\PYZus{}mm}\PY{p}{)}
\PY{n}{rule4\PYZus{}3} \PY{o}{=} \PY{n}{ctrl}\PY{o}{.}\PY{n}{Rule}\PY{p}{(}\PY{n}{th\PYZus{}hh} \PY{o}{\PYZam{}} \PY{n}{rt\PYZus{}hh}\PY{p}{,} \PY{n}{rp\PYZus{}ma}\PY{p}{)}
\PY{n}{rule4\PYZus{}4} \PY{o}{=} \PY{n}{ctrl}\PY{o}{.}\PY{n}{Rule}\PY{p}{(}\PY{n}{th\PYZus{}hh} \PY{o}{\PYZam{}} \PY{p}{(}\PY{n}{rt\PYZus{}mh} \PY{o}{|} \PY{n}{rt\PYZus{}bh}\PY{p}{)}\PY{p}{,} \PY{n}{rp\PYZus{}ba}\PY{p}{)}

\PY{c+c1}{\PYZsh{} row 5}
\PY{n}{rule5\PYZus{}1} \PY{o}{=} \PY{n}{ctrl}\PY{o}{.}\PY{n}{Rule}\PY{p}{(}\PY{n}{th\PYZus{}mh} \PY{o}{\PYZam{}} \PY{p}{(}\PY{n}{rt\PYZus{}bl} \PY{o}{|} \PY{n}{rt\PYZus{}ml}\PY{p}{)}\PY{p}{,} \PY{n}{rp\PYZus{}mc}\PY{p}{)}
\PY{n}{rule5\PYZus{}2} \PY{o}{=} \PY{n}{ctrl}\PY{o}{.}\PY{n}{Rule}\PY{p}{(}\PY{n}{th\PYZus{}mh} \PY{o}{\PYZam{}} \PY{n}{rt\PYZus{}ll}\PY{p}{,} \PY{n}{rp\PYZus{}mm}\PY{p}{)}
\PY{n}{rule5\PYZus{}3} \PY{o}{=} \PY{n}{ctrl}\PY{o}{.}\PY{n}{Rule}\PY{p}{(}\PY{n}{th\PYZus{}mh} \PY{o}{\PYZam{}} \PY{p}{(}\PY{n}{rt\PYZus{}ze} \PY{o}{|} \PY{n}{rt\PYZus{}hh}\PY{p}{)}\PY{p}{,} \PY{n}{rp\PYZus{}ma}\PY{p}{)}
\PY{n}{rule5\PYZus{}4} \PY{o}{=} \PY{n}{ctrl}\PY{o}{.}\PY{n}{Rule}\PY{p}{(}\PY{n}{th\PYZus{}mh} \PY{o}{\PYZam{}} \PY{p}{(}\PY{n}{rt\PYZus{}mh} \PY{o}{|} \PY{n}{rt\PYZus{}bh}\PY{p}{)}\PY{p}{,} \PY{n}{rp\PYZus{}ba}\PY{p}{)}

\PY{c+c1}{\PYZsh{} Контроллер}
\PY{n}{profile\PYZus{}ctrl} \PY{o}{=} \PY{n}{ctrl}\PY{o}{.}\PY{n}{ControlSystem}\PY{p}{(}\PY{p}{[}
    \PY{n}{rule1\PYZus{}1}\PY{p}{,} 
    \PY{n}{rule1\PYZus{}2}\PY{p}{,} 
    \PY{n}{rule1\PYZus{}3}\PY{p}{,} 
    \PY{n}{rule2\PYZus{}1}\PY{p}{,} 
    \PY{n}{rule2\PYZus{}2}\PY{p}{,} 
    \PY{n}{rule2\PYZus{}3}\PY{p}{,} 
    \PY{n}{rule2\PYZus{}4}\PY{p}{,} 
    \PY{n}{rule3\PYZus{}1}\PY{p}{,} 
    \PY{n}{rule3\PYZus{}2}\PY{p}{,} 
    \PY{n}{rule3\PYZus{}3}\PY{p}{,} 
    \PY{n}{rule3\PYZus{}4}\PY{p}{,} 
    \PY{n}{rule3\PYZus{}5}\PY{p}{,} 
    \PY{n}{rule4\PYZus{}1}\PY{p}{,} 
    \PY{n}{rule4\PYZus{}2}\PY{p}{,} 
    \PY{n}{rule4\PYZus{}3}\PY{p}{,} 
    \PY{n}{rule4\PYZus{}4}\PY{p}{,} 
    \PY{n}{rule5\PYZus{}1}\PY{p}{,} 
    \PY{n}{rule5\PYZus{}2}\PY{p}{,} 
    \PY{n}{rule5\PYZus{}3}\PY{p}{,} 
    \PY{n}{rule5\PYZus{}4}\PY{p}{]}\PY{p}{)}
\end{Verbatim}
\end{tcolorbox}

    \section{Тестовый пример
1}\label{ux442ux435ux441ux442ux43eux432ux44bux439-ux43fux440ux438ux43cux435ux440-1}

Клиент при заполнении анкеты выбрал следующие ответы на вопросы:

\subsection{1.1 Срок
инвестирования}\label{ux441ux440ux43eux43a-ux438ux43dux432ux435ux441ux442ux438ux440ux43eux432ux430ux43dux438ux44f}

\begin{itemize}
\tightlist
\item[$\boxtimes$]
  менее одного года (1 балл)
\item[$\square$]
  3-5 лет (3 балла)
\item[$\square$]
  6-10 лет (7 баллов)
\item[$\square$]
  11 лет и более (10 баллов)
\end{itemize}

\subsection{1.2 Срок начала использования инвестиционных
накоплений}\label{ux441ux440ux43eux43a-ux43dux430ux447ux430ux43bux430-ux438ux441ux43fux43eux43bux44cux437ux43eux432ux430ux43dux438ux44f-ux438ux43dux432ux435ux441ux442ux438ux446ux438ux43eux43dux43dux44bux445-ux43dux430ux43aux43eux43fux43bux435ux43dux438ux439}

\begin{itemize}
\tightlist
\item[$\square$]
  менее чем через 2 года (0 баллов)
\item[$\square$]
  2-5 лет (1 балл)
\item[$\boxtimes$]
  6-10 лет (4 балла)
\item[$\square$]
  11 лет и более (8 баллов)
\end{itemize}

\subsection{2.1 Уровень знаний на финансовом
рынке}\label{ux443ux440ux43eux432ux435ux43dux44c-ux437ux43dux430ux43dux438ux439-ux43dux430-ux444ux438ux43dux430ux43dux441ux43eux432ux43eux43c-ux440ux44bux43dux43aux435}

\begin{itemize}
\tightlist
\item[$\square$]
  нет знаний (0 баллов)
\item[$\square$]
  есть базовые знания (3 балла)
\item[$\boxtimes$]
  опытный инвестор (7 баллов)
\item[$\square$]
  эксперт в инвестициях (10 баллов)
\end{itemize}

\subsection{2.2 Цель
инвестиций}\label{ux446ux435ux43bux44c-ux438ux43dux432ux435ux441ux442ux438ux446ux438ux439}

\begin{itemize}
\tightlist
\item[$\square$]
  сохранение капитала и накоплений (1 балл)
\item[$\boxtimes$]
  пенсионные накопления (обеспеченное будущее) (5 баллов)
\item[$\square$]
  накопление для определенных целей (крупных покупок) (7 баллов)
\item[$\square$]
  получение максимального дохода (10 баллов)
\end{itemize}

\subsection{2.3 Ожидаемая доходность от
инвестиций}\label{ux43eux436ux438ux434ux430ux435ux43cux430ux44f-ux434ux43eux445ux43eux434ux43dux43eux441ux442ux44c-ux43eux442-ux438ux43dux432ux435ux441ux442ux438ux446ux438ux439}

\begin{itemize}
\tightlist
\item[$\square$]
  хотя бы немного выше депозита (1 балл)
\item[$\boxtimes$]
  около 10\% в год (3 балла)
\item[$\square$]
  около 20\% в год (7 баллов)
\item[$\square$]
  выше 20\% в год (10 баллов)
\end{itemize}

\subsection{2.4 Отношение к рискам
инвестирования}\label{ux43eux442ux43dux43eux448ux435ux43dux438ux435-ux43a-ux440ux438ux441ux43aux430ux43c-ux438ux43dux432ux435ux441ux442ux438ux440ux43eux432ux430ux43dux438ux44f}

\begin{itemize}
\tightlist
\item[$\square$]
  главное сохранить сбережения, доходность не так важна (0 баллов)
\item[$\boxtimes$]
  хотелось бы иметь доходность немного выше выше, чем по банковским
  вкладам (3 балла)
\item[$\square$]
  хотелось бы иметь доходность, превыщающую банковский вклад в 2 раза (7
  баллов)
\item[$\square$]
  готов инвестировать в инструменты, приносящие максимальный доход (10
  баллов)
\end{itemize}

Скоринг по результатам анкетирования:

\begin{itemize}
\tightlist
\item
  Time horizon score = 1 + 4 = 5
\item
  Risk tolerance score = 7 + 5 + 3 + 3 = 18
\end{itemize}

    \begin{tcolorbox}[breakable, size=fbox, boxrule=1pt, pad at break*=1mm,colback=cellbackground, colframe=cellborder]
\prompt{In}{incolor}{4}{\boxspacing}
\begin{Verbatim}[commandchars=\\\{\}]
\PY{c+c1}{\PYZsh{} Ввод данных}
\PY{n}{profiling} \PY{o}{=} \PY{n}{ctrl}\PY{o}{.}\PY{n}{ControlSystemSimulation}\PY{p}{(}\PY{n}{profile\PYZus{}ctrl}\PY{p}{)}
\PY{n}{profiling}\PY{o}{.}\PY{n}{input}\PY{p}{[}\PY{l+s+s1}{\PYZsq{}}\PY{l+s+s1}{time horizon score}\PY{l+s+s1}{\PYZsq{}}\PY{p}{]} \PY{o}{=} \PY{l+m+mi}{5}
\PY{n}{profiling}\PY{o}{.}\PY{n}{input}\PY{p}{[}\PY{l+s+s1}{\PYZsq{}}\PY{l+s+s1}{risk tolerance score}\PY{l+s+s1}{\PYZsq{}}\PY{p}{]} \PY{o}{=} \PY{l+m+mi}{18}

\PY{c+c1}{\PYZsh{} Вычисление типа инвестиционного профиля}
\PY{n}{profiling}\PY{o}{.}\PY{n}{compute}\PY{p}{(}\PY{p}{)}
\PY{n}{risk\PYZus{}profile}\PY{o}{.}\PY{n}{view}\PY{p}{(}\PY{n}{sim}\PY{o}{=}\PY{n}{profiling}\PY{p}{)}
\end{Verbatim}
\end{tcolorbox}

    \begin{center}
    \adjustimage{max size={0.9\linewidth}{0.9\paperheight}}{output_5_0.png}
    \end{center}
    { \hspace*{\fill} \\}
    
    \begin{tcolorbox}[breakable, size=fbox, boxrule=1pt, pad at break*=1mm,colback=cellbackground, colframe=cellborder]
\prompt{In}{incolor}{5}{\boxspacing}
\begin{Verbatim}[commandchars=\\\{\}]
\PY{n}{score} \PY{o}{=} \PY{n}{profiling}\PY{o}{.}\PY{n}{output}\PY{p}{[}\PY{l+s+s2}{\PYZdq{}}\PY{l+s+s2}{risk profile}\PY{l+s+s2}{\PYZdq{}}\PY{p}{]}
\PY{n+nb}{print}\PY{p}{(}\PY{l+s+sa}{f}\PY{l+s+s1}{\PYZsq{}}\PY{l+s+s1}{Risk profile score: }\PY{l+s+si}{\PYZob{}}\PY{n}{score}\PY{l+s+si}{:}\PY{l+s+s1}{7.4f}\PY{l+s+si}{\PYZcb{}}\PY{l+s+s1}{\PYZsq{}}\PY{p}{,} \PY{p}{)}

\PY{n+nb}{print}\PY{p}{(}\PY{l+s+s1}{\PYZsq{}}\PY{l+s+s1}{Risk profile membership:}\PY{l+s+s1}{\PYZsq{}}\PY{p}{)}
\PY{k}{for} \PY{n}{term\PYZus{}key}\PY{p}{,} \PY{n}{term\PYZus{}val} \PY{o+ow}{in} \PY{n}{risk\PYZus{}profile}\PY{o}{.}\PY{n}{terms}\PY{o}{.}\PY{n}{items}\PY{p}{(}\PY{p}{)}\PY{p}{:}
    \PY{n}{value} \PY{o}{=} \PY{n}{term\PYZus{}val}\PY{o}{.}\PY{n}{membership\PYZus{}value}\PY{p}{[}\PY{n}{profiling}\PY{p}{]}
    \PY{n+nb}{print}\PY{p}{(}\PY{l+s+sa}{f}\PY{l+s+s1}{\PYZsq{}}\PY{l+s+s1}{\PYZhy{} }\PY{l+s+si}{\PYZob{}}\PY{n}{term\PYZus{}key}\PY{l+s+si}{\PYZcb{}}\PY{l+s+s1}{: }\PY{l+s+si}{\PYZob{}}\PY{n}{value}\PY{l+s+si}{:}\PY{l+s+s1}{7.4f}\PY{l+s+si}{\PYZcb{}}\PY{l+s+s1}{ }\PY{l+s+s1}{\PYZsq{}}\PY{p}{)}
\end{Verbatim}
\end{tcolorbox}

    \begin{Verbatim}[commandchars=\\\{\}]
Risk profile score: 43.3562
Risk profile membership:
- conservative:  0.0000
- moderately-conservative:  0.2308
- moderate:  0.7692
- moderately-aggressive:  0.0000
- aggressive:  0.0000
    \end{Verbatim}

    \subsection{Результат
расчетов:}\label{ux440ux435ux437ux443ux43bux44cux442ux430ux442-ux440ux430ux441ux447ux435ux442ux43eux432}

\begin{itemize}
\tightlist
\item
  Риск-профиль клиента: умеренный
\item
  Уровень инвестиционного риска: 43.3562
\end{itemize}

    \section{Тестовый пример
2}\label{ux442ux435ux441ux442ux43eux432ux44bux439-ux43fux440ux438ux43cux435ux440-2}

Клиент при заполнении анкеты выбрал следующие ответы на вопросы:

\subsection{1.1 Срок
инвестирования}\label{ux441ux440ux43eux43a-ux438ux43dux432ux435ux441ux442ux438ux440ux43eux432ux430ux43dux438ux44f}

\begin{itemize}
\tightlist
\item[$\square$]
  менее одного года (1 балл)
\item[$\square$]
  3-5 лет (3 балла)
\item[$\boxtimes$]
  6-10 лет (7 баллов)
\item[$\square$]
  11 лет и более (10 баллов)
\end{itemize}

\subsection{1.2 Срок начала использования инвестиционных
накоплений}\label{ux441ux440ux43eux43a-ux43dux430ux447ux430ux43bux430-ux438ux441ux43fux43eux43bux44cux437ux43eux432ux430ux43dux438ux44f-ux438ux43dux432ux435ux441ux442ux438ux446ux438ux43eux43dux43dux44bux445-ux43dux430ux43aux43eux43fux43bux435ux43dux438ux439}

\begin{itemize}
\tightlist
\item[$\square$]
  менее чем через 2 года (0 баллов)
\item[$\square$]
  2-5 лет (1 балл)
\item[$\boxtimes$]
  6-10 лет (4 балла)
\item[$\square$]
  11 лет и более (8 баллов)
\end{itemize}

\subsection{2.1 Уровень знаний на финансовом
рынке}\label{ux443ux440ux43eux432ux435ux43dux44c-ux437ux43dux430ux43dux438ux439-ux43dux430-ux444ux438ux43dux430ux43dux441ux43eux432ux43eux43c-ux440ux44bux43dux43aux435}

\begin{itemize}
\tightlist
\item[$\square$]
  нет знаний (0 баллов)
\item[$\square$]
  есть базовые знания (3 балла)
\item[$\boxtimes$]
  опытный инвестор (7 баллов)
\item[$\square$]
  эксперт в инвестициях (10 баллов)
\end{itemize}

\subsection{2.2 Цель
инвестиций}\label{ux446ux435ux43bux44c-ux438ux43dux432ux435ux441ux442ux438ux446ux438ux439}

\begin{itemize}
\tightlist
\item[$\square$]
  сохранение капитала и накоплений (1 балл)
\item[$\square$]
  пенсионные накопления (обеспеченное будущее) (5 баллов)
\item[$\boxtimes$]
  накопление для определенных целей (крупных покупок) (7 баллов)
\item[$\square$]
  получение максимального дохода (10 баллов)
\end{itemize}

\subsection{2.3 Ожидаемая доходность от
инвестиций}\label{ux43eux436ux438ux434ux430ux435ux43cux430ux44f-ux434ux43eux445ux43eux434ux43dux43eux441ux442ux44c-ux43eux442-ux438ux43dux432ux435ux441ux442ux438ux446ux438ux439}

\begin{itemize}
\tightlist
\item[$\square$]
  хотя бы немного выше депозита (1 балл)
\item[$\square$]
  около 10\% в год (3 балла)
\item[$\boxtimes$]
  около 20\% в год (7 баллов)
\item[$\square$]
  выше 20\% в год (10 баллов)
\end{itemize}

\subsection{2.4 Отношение к рискам
инвестирования}\label{ux43eux442ux43dux43eux448ux435ux43dux438ux435-ux43a-ux440ux438ux441ux43aux430ux43c-ux438ux43dux432ux435ux441ux442ux438ux440ux43eux432ux430ux43dux438ux44f}

\begin{itemize}
\tightlist
\item[$\square$]
  главное сохранить сбережения, доходность не так важна (0 баллов)
\item[$\square$]
  хотелось бы иметь доходность немного выше выше, чем по банковским
  вкладам (3 балла)
\item[$\boxtimes$]
  хотелось бы иметь доходность, превыщающую банковский вклад в 2 раза (7
  баллов)
\item[$\square$]
  готов инвестировать в инструменты, приносящие максимальный доход (10
  баллов)
\end{itemize}

Скоринг по результатам анкетирования:

\begin{itemize}
\tightlist
\item
  Time horizon score = 7 + 4 = 11
\item
  Risk tolerance score = 7 + 7 + 7 + 7 = 28
\end{itemize}

    \begin{tcolorbox}[breakable, size=fbox, boxrule=1pt, pad at break*=1mm,colback=cellbackground, colframe=cellborder]
\prompt{In}{incolor}{6}{\boxspacing}
\begin{Verbatim}[commandchars=\\\{\}]
\PY{c+c1}{\PYZsh{} Ввод данных}
\PY{n}{profiling} \PY{o}{=} \PY{n}{ctrl}\PY{o}{.}\PY{n}{ControlSystemSimulation}\PY{p}{(}\PY{n}{profile\PYZus{}ctrl}\PY{p}{)}
\PY{n}{profiling}\PY{o}{.}\PY{n}{input}\PY{p}{[}\PY{l+s+s1}{\PYZsq{}}\PY{l+s+s1}{time horizon score}\PY{l+s+s1}{\PYZsq{}}\PY{p}{]} \PY{o}{=} \PY{l+m+mi}{11}
\PY{n}{profiling}\PY{o}{.}\PY{n}{input}\PY{p}{[}\PY{l+s+s1}{\PYZsq{}}\PY{l+s+s1}{risk tolerance score}\PY{l+s+s1}{\PYZsq{}}\PY{p}{]} \PY{o}{=} \PY{l+m+mi}{28}

\PY{c+c1}{\PYZsh{} Вычисление типа инвестиционного профиля}
\PY{n}{profiling}\PY{o}{.}\PY{n}{compute}\PY{p}{(}\PY{p}{)}
\PY{n}{risk\PYZus{}profile}\PY{o}{.}\PY{n}{view}\PY{p}{(}\PY{n}{sim}\PY{o}{=}\PY{n}{profiling}\PY{p}{)}
\end{Verbatim}
\end{tcolorbox}

    \begin{center}
    \adjustimage{max size={0.9\linewidth}{0.9\paperheight}}{output_9_0.png}
    \end{center}
    { \hspace*{\fill} \\}
    
    \begin{tcolorbox}[breakable, size=fbox, boxrule=1pt, pad at break*=1mm,colback=cellbackground, colframe=cellborder]
\prompt{In}{incolor}{7}{\boxspacing}
\begin{Verbatim}[commandchars=\\\{\}]
\PY{n}{score} \PY{o}{=} \PY{n}{profiling}\PY{o}{.}\PY{n}{output}\PY{p}{[}\PY{l+s+s2}{\PYZdq{}}\PY{l+s+s2}{risk profile}\PY{l+s+s2}{\PYZdq{}}\PY{p}{]}
\PY{n+nb}{print}\PY{p}{(}\PY{l+s+sa}{f}\PY{l+s+s1}{\PYZsq{}}\PY{l+s+s1}{Risk profile score: }\PY{l+s+si}{\PYZob{}}\PY{n}{score}\PY{l+s+si}{:}\PY{l+s+s1}{7.4f}\PY{l+s+si}{\PYZcb{}}\PY{l+s+s1}{\PYZsq{}}\PY{p}{,} \PY{p}{)}

\PY{n+nb}{print}\PY{p}{(}\PY{l+s+s1}{\PYZsq{}}\PY{l+s+s1}{Risk profile membership:}\PY{l+s+s1}{\PYZsq{}}\PY{p}{)}
\PY{k}{for} \PY{n}{term\PYZus{}key}\PY{p}{,} \PY{n}{term\PYZus{}val} \PY{o+ow}{in} \PY{n}{risk\PYZus{}profile}\PY{o}{.}\PY{n}{terms}\PY{o}{.}\PY{n}{items}\PY{p}{(}\PY{p}{)}\PY{p}{:}
    \PY{n}{value} \PY{o}{=} \PY{n}{term\PYZus{}val}\PY{o}{.}\PY{n}{membership\PYZus{}value}\PY{p}{[}\PY{n}{profiling}\PY{p}{]}
    \PY{n+nb}{print}\PY{p}{(}\PY{l+s+sa}{f}\PY{l+s+s1}{\PYZsq{}}\PY{l+s+s1}{\PYZhy{} }\PY{l+s+si}{\PYZob{}}\PY{n}{term\PYZus{}key}\PY{l+s+si}{\PYZcb{}}\PY{l+s+s1}{: }\PY{l+s+si}{\PYZob{}}\PY{n}{value}\PY{l+s+si}{:}\PY{l+s+s1}{7.4f}\PY{l+s+si}{\PYZcb{}}\PY{l+s+s1}{ }\PY{l+s+s1}{\PYZsq{}}\PY{p}{)}
\end{Verbatim}
\end{tcolorbox}

    \begin{Verbatim}[commandchars=\\\{\}]
Risk profile score: 75.8094
Risk profile membership:
- conservative:  0.0000
- moderately-conservative:  0.0000
- moderate:  0.0000
- moderately-aggressive:  0.5000
- aggressive:  0.3077
    \end{Verbatim}

    \subsection{Результат
расчетов:}\label{ux440ux435ux437ux443ux43bux44cux442ux430ux442-ux440ux430ux441ux447ux435ux442ux43eux432}

\begin{itemize}
\tightlist
\item
  Риск-профиль клиента: умеренно-аггресивный
\item
  Уровень инвестиционного риска: 75.8094
\end{itemize}

    \section{Список ссылок на
источники}\label{ux441ux43fux438ux441ux43eux43a-ux441ux441ux44bux43bux43eux43a-ux43dux430-ux438ux441ux442ux43eux447ux43dux438ux43aux438}

\begin{itemize}
\tightlist
\item
  \href{https://openjournals.libs.uga.edu/fsr/article/view/3890}{Financial
  risk tolerance revisited -
  https://openjournals.libs.uga.edu/fsr/article/view/3890}
\item
  \href{https://www.schwab.com/resource/investment-questionnaire}{Charles
  Schwab: Investor Profile Questionnaire -
  https://www.schwab.com/resource/investment-questionnaire}
\item
  \href{https://www.morningstar.com/views/blog/risk/how-to-measure-investment-risk-profile}{Do
  Investment Risk Profiles Work? How to Calculate Profiles for Top-Notch
  Financial Plans -
  https://www.morningstar.com/views/blog/risk/how-to-measure-investment-risk-profile}
\end{itemize}


    % Add a bibliography block to the postdoc
    
    
    
\end{document}
